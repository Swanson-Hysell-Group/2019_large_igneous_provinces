\documentclass[11pt,letterpaper]{article}

%\usepackage{fontspec}
%\usepackage[utf8]{inputenc}
\usepackage{textcomp,marvosym}
\usepackage{amsmath,amssymb}
\usepackage[normalem]{ulem}
\usepackage[left]{lineno}
\usepackage{changepage}
\usepackage{rotating}
\usepackage{color}
\usepackage{natbib}
\usepackage{setspace}
\usepackage{}
\usepackage{fancyhdr}
\usepackage{graphicx}
\usepackage{xspace}
\usepackage[hidelinks]{hyperref}
\urlstyle{same}
\usepackage{threeparttable}
\doublespacing

\raggedright
\textwidth = 6.5 in
\textheight = 8.25 in
\oddsidemargin = 0.0 in
\evensidemargin = 0.0 in
\topmargin = 0.0 in
\headheight = 0.0 in
\headsep = 0.5 in
\parskip = 0.1 in
\parindent = 0.2in

% Bold the 'Figure #' in the caption and separate it from the title/caption with a period
% Captions will be left justified
\usepackage[aboveskip=1pt,labelfont=bf,labelsep=period,justification=raggedright,singlelinecheck=off]{caption}

% Remove brackets from numbering in List of References
%\makeatletter
%\renewcommand{\@biblabel}[1]{\quad#1.}
%\makeatother

% Self defined commands
\newcommand{\degC}{$^{\circ}$C\xspace}
\newcommand{\dC}{$\delta^{13}$C\xspace}
\newcommand{\dO}{$\delta^{18}$O\xspace}
\newcommand{\SrSr}{$^{87}$Sr/$^{86}$Sr\xspace}
\newcommand{\permil}{\textperthousand\xspace}
\newcommand{\dsil}{$d$\xspace}
\newcommand{\UPb}{$^{206}$Pb/$^{238}$U\xspace}
%

\pagestyle{myheadings}
\pagestyle{fancy}
\fancyhf{}
\lhead{Park et al., in preparation for XXX}
\rhead{\thepage}

\begin{document}

\begin{flushleft}
{\Large \textbf{XXX}}
\\
Yuem Park\textsuperscript{1},
Nicholas L. Swanson-Hysell\textsuperscript{1}
Francis M. Macdonald\textsuperscript{1}
\\
\bigskip
\textsuperscript{1} Department of Earth and Planetary Science, University of California, Berkeley, CA, USA
\textsuperscript{1} Department of Earth Science, University of California, Santa Barbara, CA, USA
\\
\textsuperscript{2} XXX
\bigskip

\end{flushleft}

\noindent\textit{This article is in preparation for submission to XXX.}

\linenumbers

\section*{ABSTRACT \label{sec:ABSTRACT}}

XXX

\section*{INTRODUCTION \label{sec:INTRODUCTION}}

The proposed environmental impacts of large igneous province (LIP) eruptions are myriad (as summarized in \citealp{Ernst2017}). One aspect of LIP emplacement that has been hypothesized to relate to long-term global climate is the effect such emplacement could have on increasing global weatherability and driving cooling. Global weatherability is the sum of factors aside from climate itself that contribute to overall global weathering. On a planet with high weatherability, the CO$_2$ input from volcanism can be removed via silicate weathering at a lower CO$_2$ value than on a less weatherable planet. The relatively high concentrations of Ca and Mg (that ultimately sequester carbon through precipitation as carbonate) and the high weathering rates of mafic lithologies make it so that basaltic regions consume relatively more CO$_2$ than regions where bedrock composition is closer to bulk continental crust \citep{Dessert2003a}. Data from basaltic watersheds show that chemical weathering rates are greatest in regions with high runoff and as a result CO$_2$ consumption in basaltic regions is most pronounced in the tropical rain belt \citep{Dessert2003a, Hartmann2014a}.

Given these factors, the emplacement of LIPs in the tropics has been associated with specific episodes of climatic cooling on Earth. In the Neoproterozoic, the emplacement of the Franklin LIP in the tropics ca. 720 Ma, in concert with elevated runoff rates associated with continental break-up, has been implicated as a major contributor to the cooling that initiated the Sturtian snowball Earth \citep{Donnadieu2004b, Cox2016}. In the Cenozoic, the movement of the Deccan LIP into the tropical rain belt has been implicated in drawing down CO$_2$ levels in the lead-up to Eocene glaciation of Antarctica \citep{Kent2008a}.

%Kent2008a 10.1073��pnas.0805382105

This chapter seeks to address two interconnected questions: 1) how strong is the relationship between tropical LIP area and glaciation? and 2) how unique are the proposed events in terms of low-latitude LIP area peaks?

\section*{METHODS}

Outlines of large igneous provinces (LIPs) through the Phanerozoic (Fig. \ref{fig:LIP_map}) were taken from the compilation of \cite{Ernst2017a}, and emplacement ages were taken from the literature (Table \ref{tab:LIPs}). The compilation of \cite{Ernst2017a} seeks to reconstruct the original areal extent of LIPs with the caveat that there can be significant uncertainty with doing so, particularly for older more deeply eroded LIPs. These polygons encapsulate all of the rocks associated with a given LIP including dikes and sills. For some LIPs, this approach may lead to an overestimate of original areal extent given that subsurface intrusions could extend over a broader area than the surface volcanics. However, such intrusions likely provide the best estimate available of original extent for ancient LIPs. The extent of presently exposed volcanics associated with the LIPs were taken from a number of resources including the PLATES compilation (Table \ref{tab:LIPs}; \citealp{Coffin2006a}) and more recent compilation efforts associated with the Large Igneous Provinces Commission. 

After LIPs are emplaced they will progressively erode. In order to account for this decrease in area through with time \cite{Godderis2017a} took the approach of fitting estimates of the original surface extent and estimates of the current extent for five LIPs with an exponential decay function. They used the resulting exponential decay constants to develop a first-order model to estimate changing LIP area through time. We extend this approach to 19 basaltic LIPs for which we have estimates of the original areal extent of the province and the current areal extent of rocks associated with it.


The post-emplacement weathering and erosional history of each LIP should be dependent on the tectonic and climatic setting that each LIP experiences during and after emplacement. 

The original extent LIP polygons were assigned a plate ID corresponding to a tectonic unit on Earth using the polygons of \cite{Torsvik2016a} for the Phanerozoic. The LIP polygons were reconstructed from 520 Ma to the present along with the tectonic units utilizing the paleogeographic model of \cite{Torsvik2016a} in the spin axis reference frame (anchor plate ID of 1). This paleogeographic model was updated to include a revision to Ordovician Laurentia \citep{Swanson-Hysell2017a} and the Paleozoic of Asia \citep{Domeier2018a}. Reconstructions and area calculations within latitude bands utilized the pyGPlates function library and custom Python scripts documented within a Jupyter notebook that reproduces the analysis and development of the associated visualizations. 





For example, if a LIP is emplaced into an actively rifting basin and is quickly buried by sediments, its ability to consume CO$_2$ through silicate weathering could be significantly attenuated soon after burial. In a similar manner, without active uplift, soil shielding from regolith development on low-relief LIPs could significantly decrease the local weatherability of a LIP and mute its impact on global weatherability (9). Furthermore, the thickness of this regolith is dependent on the local climate (185). However, it is difficult to constrain when and how much each of these factors affect the post-emplacement weathering and erosional history of each LIP—in many cases the timing and extent of burial of LIPs emplaced into actively rifting basins is poorly constrained, and the timing, extent, and effect of regolith development on LIP weatherability is difficult to model. Ultimately, however, we expect all LIPs to effectively cease consuming appreciable atmospheric CO2 at some point after emplacement, either through burial or consumption of the majority of weatherable material.
In order to account for these post-emplacement weathering and erosional histories, we imposed two post-emplacement scenarios on our calculated LIP areas (Fig. S4). In the “decay” scenario, the areas of all LIPs decay exponentially after emplacement. This approach follows that of Goddéris et al. (186) wherein the original and present-day area of five Phanerozoic LIPs were compared and exponential decay was assumed. In the analysis of Goddéris et al. (186) half-lives of the LIPs were observed to be between 90 to more than 400 Myr, with three of the five half-lives clustering at ~100 Myr. In our analysis, we take a half-life of 100 Myr as reasonable to first order, and use this value as the exponential decay half-life of the LIPs. In the “decay+burial” scenario, LIPs associated with successful rifting, associated thermal subsidence, and geological evidence for at least partial sedimentary cover soon after emplacement were instantly removed from the paleogeographic reconstruction 5 Myrs after emplacement, while all other LIPs were treated the same as in the “decay” scenario. In a successful rift, the timescale of burial of extrusive volcanics by sediments due to subsidence may vary from 0 to ~10 Myr depending on relative sea level, geometry of the rift and associated volcanics, and timing of emplacement relative to the rift-drift transition. Thus, we use 5 Myrs as a conservative estimate for timescale of burial after emplacement. The LIPs deemed to be associated with successful rifting and subsidence are: CAMP, Madagascar, Maud Rise, North Atlantic Igneous Province, NW Australia, Parana-Etendeka, Seychelles, and Wichita (Table S2).
Our LIP reconstruction differs from that of Johansson et al. (29) in several respects. In contrast to the decay and decay+burial scenarios, Johansson et al. (29) uses a static extent for each LIP throughout the reconstruction with some of the polygons corresponding to the present-day surface extent and some of the representing the original extent that includes currently buried portions of the LIP (e.g. the North American Midcontinent Rift). The LIP compilation we utilize seeks to outline the original extent of the LIPs through use of the area of associated dikes and sills. The LIP reconstruction used in this study includes pre-Phanerozoic LIPs. However, given the imposed exponential decay since emplacement, the inclusion of these LIPs do not significantly impact the calculated LIP areas, as can be seen by the small area of LIPs prior to ca. 380 Ma in Fig. S4.
In the “decay” scenario, we observe two primary peaks and one minor peak in calculated LIP area within the tropics (Fig. S4B). The first primary peak starting ca. 377 Ma is associated with the emplacement of the Kola-Dnieper and the second primary peak starting ca. 251 Ma is associated with the emplacement of the Central Atlantic magmatic province (CAMP). A Cenozoic peak is associated with both the ca. 30 Ma emplacement of the Afro-Arabian (Afar) LIP as well as the effect of the earlier drift of the Deccan LIP into the tropics. In the “burial” scenario, the peak associated with the CAMP is transient, but the other peaks remain (Fig. S4).

\section*{RESULTS}

\section*{DISCUSSION}

In the models that proposed the ``Fire and Ice'' hypothesis for the Sturtian glaciation, chemical weathering was modeled as an Arhennius relationship where it was solely a function of temperature and run-off \citep{Donnadieu2004a}. However, such an approach neglects the effects that can result through soil shielding and regolith development in low relief regions. Recent progress on understanding the relationships between landscapes and chemical weathering (e.g. \cite{Maher2014a}) reveals that these effects are quite important. As a result, more recent modeling of chemical weathering that has incorporated such processes highlights the importance of high-relief regions \citep{Godderis2017a} (NOT SURE IF RIGHT REF FOR 2017 there are a few).

\section*{CONCLUSIONS}

\section*{CONCLUSIONS}

\section*{ACKNOWLEDGEMENTS \label{sec:ACKNOWLEDGEMENTS}}


\section*{FIGURES}

LIP Location map
Panel A) 0 to 500 Ma
Panel B) 500 to 1200 Ma


LIP fractional area remaining with time

LIP area through the tropics through time
Panel A) ice extent
Panel B) all latitude 
Panel C) in the tropics (different models)
Panel D) heat map of one of the models

Monte Carlo null hypothesis test
Panel A) total LIP
Panel B) LIP in the tropics
Panel C) continental fraction in the tropics
Panel D) sutures(??)



\footnotesize

XXX

\singlespacing

\bibliographystyle{gsabull}
\bibliography{References}

\end{document}
